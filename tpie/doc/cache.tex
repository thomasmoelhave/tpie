%%
%% File: cache.tex
%% Author: Octavian Procopiuc <tavi@cs.duke.edu>
%% Created: Sep 26, 2000
%%
%% Last updated on: Sep 26, 2000
%% by: Octavian Procopiuc <tavi@cs.duke.edu>
%%

\documentclass[11pt]{article}

\usepackage{fullpage}
\usepackage{verbatim}
\usepackage{epsfig}
%\usepackage{times}

\begin{document}

\title{\bf The {\tt CACHE\_MANAGER} Class}
\date{Draft of \today}
\author{}
\maketitle

\section{Overview}
\label{cache:overview}

\section{Class Declaration}

   \begin{tabbing}
   \hspace*{.3in} \= \hspace{.5in} \= \\

   \> {\tt template<class T, class W> class CACHE\_MANAGER;}
   \end{tabbing}

\section{Constructors and Destructor}
   \begin{tabbing}
   \hspace*{.3in} \= \hspace{.5in} \= \\

   \> {\tt CACHE\_MANAGER(size\_t capacity)}\\ 
   \>\>\parbox[t]{5.5in}{Construct a fully-associative cache manager with the given capacity.}\\[3mm]

   \> {\tt CACHE\_MANAGER(size\_t capacity, size\_t assoc)}\\ 
   \>\>\parbox[t]{5.5in}{Construct a cache manager with the given capacity and associativity.}\\[3mm]

   \> {\tt \verb|~|CACHE\_MANAGER()}\\ 
   \>\>\parbox[t]{5.5in}{Destructor. Write out all items still in the cache.}\\[3mm]

   \end{tabbing}

\section{Member functions}
   \begin{tabbing}
   \hspace*{.3in} \= \hspace{.5in} \= \\ 

   \> {\tt  bool read(size\_t k, T \& item);}\\ 
   \>\>\parbox[t]{5.5in}{Read an item from the cache based on key {\tt k} and store it in {\tt item}. If found, the item is removed from the cache. Return true if the key was found.}\\[3mm]

   \> {\tt  bool write(size\_t k, const T \& item);}\\ 
   \>\>\parbox[t]{5.5in}{Write an item in the cache based on the given key {\tt k}. If the cache was full, the least recently used item is writen out using the {\tt W} function object, and it is removed from the cache.}\\[3mm]

   \> {\tt  bool erase(size\_t k);}\\ 
   \>\>\parbox[t]{5.5in}{Erase an item from the cache based on the given key {\tt k}. Return true if the key was found.}\\[3mm]
   \end{tabbing}

\section{Implementation Details}

The cache is an array of pairs (key, item of type {\tt T}). The array is divided into (capacity/assoc) sets, and inside a set item are inserted in position 0 and deleted from the last position. This insures LRU behavior.
\end{document}
